\documentclass[13pt]{amsart}

\usepackage{amsfonts,latexsym,amsthm,amssymb,amsmath,amscd,euscript}
\usepackage{fullpage}
\usepackage[margin=0.5in]{geometry}
\usepackage{hyperref}
\usepackage{mathtools}
\usepackage{charter}
\usepackage{natbib}

\usepackage[dvipsnames]{xcolor}
\usepackage[usenames,dvipsnames]{pstricks}

\usepackage{hyperref}
\hypersetup{
    pdffitwindow=false,            % window fit to page
    pdfstartview={Fit},            % fits width of page to window
    pdftitle={Notes on NetworkLending},     % document title
    pdfauthor={Taylor Kessinger},         % author name
    pdfsubject={},                 % document topic(s)
    pdfnewwindow=true,             % links in new window
    colorlinks=true,               % coloured links, not boxed
    linkcolor=OrangeRed,      % colour of internal links
    citecolor=ForestGreen,       % colour of links to bibliography
    filecolor=Orchid,            % colour of file links
    urlcolor=Cerulean           % colour of external links
}

\newcommand{\B}{\mathcal{B}}
\newcommand{\C}{\mathcal{C}}

\begin{document}

This is a series of notes on \texttt{NetworkLending}, a \texttt{Julia} package for modeling the evolution of cooperation with a central bank that lends money and expects repayment, as well as the underlying theory.
The game played is a public goods game with lending and repayment, played pairwise.
Each individual has a strategy that ranges from $0$ to $3$, with the larger ``bit'' representing whether they defect or cooperate in the PGG and the smaller ``bit'' representing whether they shirk or pay back loans.
The pairwise game proceeds as follows:
\begin{itemize}
    \item The bank lends each individual $1$ unit of capital.
    If the individual has a ``bad'' reputation, they are instead lent $1-d$ units of capital, with $d$ a decrement for being a bad actor.
    \item Cooperators invest all their money in the common pot.
    Defectors invest none.
    \item The money in the pot is multiplied by a synergy factor $r$ and evenly distributed among the players.
    \item Shirkers do not pay the bank back.
    Payersback pay the bank an amount of money $z$ (from German ``Zins'', meaning interest).
\end{itemize}
Strategies are labeled as follows:
\begin{itemize}
    \item $0$ is shirk/defect.
    \item $1$ is payback/defect.
    \item $2$ is shirk/cooperate.
    \item $3$ is payback/cooperate.
\end{itemize}

\begin{section}{Relationship between the PGG and the prisoner's dilemma}
We begin by recapitulating the relationship between the PGG and a prisoner's dilemma (PD).
In a PD, we have
\end{section}\begin{equation}
    a_{ij} =
    \begin{pmatrix}
        0 & \B \\
        \C & \B - \C
    \end{pmatrix}
\end{equation}
or, equivalently,
\begin{equation}
    a_{ij} =
    \begin{pmatrix}
        P & T \\
        S & R,
    \end{pmatrix}
\end{equation}
with $T > R > P > S$.
(Note that these labels are opposite the usual labeling of a PD, because for us, the ``smaller'' index is defection.)
In a pairwise PGG, we have
\begin{equation}
    a_{ij} =
    \begin{pmatrix}
        0 & \frac{r}{2} \\
        \frac{r}{2} - 1 & r - 1
    \end{pmatrix}.
\end{equation}
This allows us to write
\begin{equation}
    \begin{split}
        \mathcal{B} & = \frac{r}{2}\\
        \mathcal{C} & = 1 - \frac{r}{2}.
    \end{split}
\end{equation}
We can also note that the PD condition $T > R > P > S$ implies
\begin{equation}
    \begin{split}
        \mathcal{B} > \mathcal{B} - \mathcal{C} & > 0 > -\mathcal{C} \\
        \therefore \mathcal{B} & > \mathcal{C} > 0 \\
        \therefore r & > 2 - r > 0 \\
        \therefore r & > 2 - r \\
        \therefore 2r & > 2 \\
        \therefore r & > 1, \text{and}\\
        2 - r & > 0\\
        \therefore -r & > -2\\
        \therefore r & < 2.
    \end{split}
\end{equation}
In other words, $2 > r > 1$ is the bare minimum needed for our game to even \emph{be} a prisoner's dilemma.
Small wonder that, in simulations where $r > 2$, cooperation shoots through the roof.
Derp.
The $\mathcal{B}/\mathcal{C} > k$ condition implies
\begin{equation}
    \begin{split}
        \frac{r}{2-r} & > k \\
        \therefore r & > k(2-r) \\
        \therefore r & > 2k - kr \\
        \therefore r(k+1) & > 2k \\
        \therefore r & > \frac{2k}{k+1}.
    \end{split}
\end{equation}
We need $2 > 2k/(k+1)$ for cooperation to be possible, or equivalently
\begin{equation}
    \begin{split}
        \frac{2k}{k+1} & < 2 \\
        \therefore 2k & < 2k + 2 \\
        \therefore 0 & < 2,
    \end{split}
\end{equation}
which is a pretty easy condition to satisfy.
For $k = 3$ we require $r > 1.5$, and for $k = 4$ we require $r > 1.6$.
For a much higher value of $k$, say $10$, we need $r > 20/11 \approx 1.82$.

\section{The PGG with lending}

The payoff matrix for us in the case with $d = 0$ is
\begin{equation}
    a_{ij} =
        \begin{pmatrix}
        0 & 0 & \frac{r}{2} & \frac{r}{2} \\
        -z & -z & \frac{r}{2} - z & \frac{r}{2} - z \\
        \frac{r}{2} - 1 & \frac{r}{2} - 1 & r - 1 & r - 1 \\
        \frac{r}{2} - 1 - z & \frac{r}{2} - 1 - z  & r - 1 - z & r - 1 - z
    \end{pmatrix}.
\end{equation}
(Somewhat embarrassingly, preparing the rest of this document convinced me that I had actually messed up the above payoff matrix.)
It is not entirely obvious what to conclude by looking at this, but we may be able to make some headway by considering the replicator dynamics.
The replicator equation is
\begin{equation}
    \dot{x}_i = x_i (f_i + g_i - \phi),
\end{equation}
with
\begin{equation}
    \begin{split}
        f_i & = \sum_j a_{ij} x_j,\\
        g_i & = \sum_j b_{ij} x_j,\\
        b_{ij} & = \frac{(k+1)a_{ii} + a_{ij} - a_{ji} - (k+1)a_{jj}}{(k+1)(k-2)}, \text{and}\\
        \phi & = \sum_i x_i f_i.
    \end{split}
\end{equation}
The $b_{ij}$ terms are correct for death-birth updating, which we are using.
We can do the relevant algebra, filling in $b_{ij}$ first and ignoring the $\frac{1}{(k+1)(k-2)}$ prefactor:
\begin{equation}
    \begin{split}
        b_{01} & = (k+1)a_{00} + a_{01} - a_{10} - (k+1)a_{11} = z + (k+1)z \\
        b_{02} & = (k+1)a_{00} + a_{02} - a_{20} - (k+1)a_{22} = \frac{r}{2} - (\frac{r}{2} - 1) - (k+1)(r-1)\\
        b_{03} & = (k+1)a_{00} + a_{03} - a_{30} - (k+1)a_{33} = \frac{r}{2} - (\frac{r}{2} - 1 - z) - (k+1)(r - 1 - z) \\
        b_{12} & = (k+1)a_{11} + a_{12} - a_{21} - (k+1)a_{22} = (k+1)(-z) + \frac{r}{2} - z - (\frac{r}{2} - 1) - (k+1)(r-1)\\
        b_{13} & = (k+1)a_{11} + a_{13} - a_{31} - (k+1)a_{33} = (k+1)(-z) + \frac{r}{2} - z - (\frac{r}{2} - 1 - z) - (k+1)(r - 1 - z)\\
        b_{23} & = (k+1)a_{22} + a_{23} - a_{32} - (k+1)a_{33} = (k+1)(r-1) + (r-1) - (r - 1 - z) - (k+1)(r - 1 - z),
    \end{split}
\end{equation}
which is actually all that is needed to fill in $b_{ij}$, as $b_{ii} = 0 ~ \forall ~ i$ and $b_{ij} = -b_{ji}$ (the matrix is antisymmetric).
A little simplification yields
\begin{equation}
    \begin{split}
        b_{01} & = z + (k+1)z \\
        b_{02} & = 1 - (k+1)(r-1) \\
        b_{03} & = 1 + z - (k+1)(r - 1 - z) \\
        b_{12} & = (k+1)(-z) - z + 1 - (k+1)(r-1)\\
        b_{13} & = (k+1)(-z) + 1 - (k+1)(r - 1 - z)\\
        b_{23} & = (k+1)(r-1) + z - (k+1)(r - 1 - z).
    \end{split}
\end{equation}
We thus have
\begin{equation}
    \begin{split}
        f_i & =
        \begin{pmatrix}
            a_{00}x_0 + a_{01}x_1 + a_{02}x_2 + a_{03}x_3\\
            a_{10}x_0 + a_{11}x_1 + a_{12}x_2 + a_{13}x_3\\
            a_{20}x_0 + a_{21}x_1 + a_{22}x_2 + a_{23}x_3\\
            a_{30}x_0 + a_{31}x_1 + a_{32}x_2 + a_{33}x_3
        \end{pmatrix}
        \\
        & =
        \begin{pmatrix}
            \frac{r}{2}(x_2 + x_3)\\
            \frac{r}{2}(x_2 + x_3) - z\\
            (\frac{r}{2} - 1)(x_0 + x_1) + (r-1)(x_2 + x_3)\\
            (\frac{r}{2} - 1 - z)(x_0 + x_1) + (r - 1 - z)(x_2 + x_3)
        \end{pmatrix}
        \\
        & =
        \begin{pmatrix}
            \frac{r}{2}(x_2 + x_3)\\
            \frac{r}{2}(x_2 + x_3) - z\\
            \frac{r}{2}(x_0 + x_1) + \frac{r}{2} - 1\\
            \frac{r}{2}(x_0 + x_1) + \frac{r}{2} - 1 - z
        \end{pmatrix},
    \end{split}
\end{equation}
\begin{equation}
    \begin{split}
        g_i & = \frac{1}{(k+1)(k-2)}
        \begin{pmatrix}
            b_{00}x_0 + b_{01}x_1 + b_{02}x_2 + b_{03}x_3\\
            b_{10}x_0 + b_{11}x_1 + b_{12}x_2 + b_{13}x_3\\
            b_{20}x_0 + b_{21}x_1 + b_{22}x_2 + b_{23}x_3\\
            b_{30}x_0 + b_{31}x_1 + b_{32}x_2 + b_{33}x_3
        \end{pmatrix}
        \\
        & = \frac{1}{(k+1)(k-2)}
        \begin{pmatrix}
            b_{01}x_1 + b_{02}x_2 + b_{03}x_3\\
            -b_{10}x_0 + b_{12}x_2 + b_{13}x_3\\
            -b_{02}x_0 - b_{12}x_1 + b_{23}x_3\\
            -b_{03}x_0 - b_{13}x_1 - b_{23}x_2
        \end{pmatrix}
        \\
        & = \frac{1}{(k+1)(k-2)} \times
        \\
        &
        \begin{pmatrix}
            [z + (k+1)z]x_1 + [1 - (k+1)(r-1)]x_2 + [1 + z - (k+1)(r - 1 - z)]x_3\\
            -[z + (k+1)z]x_0 + [(k+1)(-z) - z + 1 - (k+1)(r-1)]x_2 + [(k+1)(-z) + 1 - (k+1)(r - 1 - z)]x_3\\
            -[1 - (k+1)(r-1)]x_0 - [(k+1)(-z) - z + 1 - (k+1)(r-1)]x_1 + [(k+1)(r-1) + z - (k+1)(r - 1 - z)]x_3\\
            -[1 + z - (k+1)(r - 1 - z)]x_0 - [(k+1)(-z) + 1 - (k+1)(r - 1 - z)]x_1 - [(k+1)(r-1) + z - (k+1)(r - 1 - z)]x_2
        \end{pmatrix},
    \end{split}
\end{equation}
and
\begin{equation}
    \begin{split}
        \phi & = x_0[\frac{r}{2}(x_2 + x_3)] + x_1[\frac{r}{2}(x_2 + x_3) - z] + x_2[\frac{r}{2}(x_0 + x_1) + \frac{r}{2} - 1] + x_3[\frac{r}{2}(x_0 + x_1) + \frac{r}{2} - 1 - z]
        \\
        & = (x_0 + x_1)[\frac{r}{2}(x_2 + x_3)] + (x_2 + x_3)[\frac{r}{2}(x_0 + x_1) + \frac{r}{2} - 1] - (x_1 + x_3)z
        \\
        & = r(x_0 + x_1)(x_2 + x_3) + (x_2 + x_3)(\frac{r}{2} - 1) - (x_1 + x_3)z.
    \end{split}
    \label{eq:phi}
\end{equation}
The last line of equation \ref{eq:phi} has a straightforward interpretation: the average fitness depends on the product of the cooperator and defector frequencies, plus an additional term corresponding to cooperator frequency, minus the amount that gets paid back each generation.
This looks like it's beyond my analytical ability, but we could certainly try to solve it numerically.

In the case where individuals are given different endowments $v_i$ and $v_j$, the total amount of money in the pot is the sum of the two, so the payoff to individual $i$ is $r\frac{v_i + v_j}{2}$.
The payoff matrix \emph{without lending} becomes
\begin{equation}
    a_{ij} =
        \begin{pmatrix}
            0 & r\frac{v_j}{2} \\
            r\frac{v_i}{2} - v_i & r\frac{v_i + v_j}{2} - v_i
        \end{pmatrix}
        =
        \begin{pmatrix}
            0 & r\frac{v_j}{2} \\
            [\frac{r}{2} - 1] v_i & [\frac{r}{2} - 1]v_i + r\frac{v_j}{2}
        \end{pmatrix}.
\end{equation}
(Without loss of generality, we could divide out by $v_i$ and reduce the number of parameters by one.)
Note that, if individual $i$ receives $v_i$ from the bank, a payerback will have to pay $zv_i$.
So the payoff matrix \emph{with lending} becomes
\begin{equation}
    a_{ij} =
    \begin{pmatrix}
        0 & 0 & r\frac{v_j}{2} & r\frac{v_j}{2} \\
        -zv_i & -zv_i & r\frac{v_j}{2} - zv_i & r\frac{v_j}{2} - zv_i \\
        r\frac{v_i}{2} - v_i & r\frac{v_i}{2} - v_i & r\frac{v_i + v_j}{2} - v_i & r\frac{v_i + v_j}{2} - v_i \\
        r\frac{v_i}{2} - (z+1)v_i & r\frac{v_i}{2} - (z+1)v_i  & r\frac{v_i + v_j}{2} - (z+1)v_i & r\frac{v_i + v_j}{2} - (z+1)v_i
    \end{pmatrix}.
\end{equation}
If we assume that reputations are updated \emph{instantly}, so that $v_i$ is $v$ for payersback (types $1$ and $3$) and $v(1-d)$ for shirkers (types $0$ and $2$), then the size of the pot is $2v$ for two payersback, $v(2-d)$ for one shirker, and $v(2 - 2d)$ for two shirkers, given that everyone cooperates.
Put another way, the pot contributions are $0, 0, 1-d, 1$.
We might want to define a ``pot matrix''
\begin{equation}
    p_{ij} =
    \begin{pmatrix}
        0 & 0 & 1-d & 1 \\
        0 & 0 & 1-d & 1 \\
        1-d & 1-d & 2-2d & 2-d \\
        1 & 1 & 2-d & 2
    \end{pmatrix},
\end{equation}
a ``payback matrix''
\begin{equation}
    l_{ij} =
    \begin{pmatrix}
        0 & 0 & 0 & 0 \\
        1 & 1 & 1 & 1 \\
        0 & 0 & 0 & 0 \\
        1 & 1 & 1 & 1
    \end{pmatrix},
\end{equation}
and an ``investment matrix''
\begin{equation}
    r_{ij} =
    \begin{pmatrix}
        0 & 0 & 0 & 0 \\
        0 & 0 & 0 & 0 \\
        1-d & 1-d & 1-d & 1-d \\
        1 & 1 & 1 & 1
    \end{pmatrix},
\end{equation}
so that the total payoff matrix becomes
\begin{equation}
    a_{ij} = v(\frac{r}{2} p_{ij} - r_{ij} - zl_{ij}).
\end{equation}
In this case, the payoff matrix is
\begin{equation}
    a_{ij} = v
    \begin{pmatrix}
        0 & 0 & \frac{r}{2}(1-d) & \frac{r}{2} \\
        -z & -z & \frac{r}{2}(1-d) -z & \frac{r}{2} -z \\
        (\frac{r}{2} - 1)(1-d) & (\frac{r}{2} - 1)(1-d) & (r-1)(1-d)& \frac{r}{2}(2 - d) - (1-d)\\
        \frac{r}{2} - 1 - z & \frac{r}{2} - 1 -z & \frac{r}{2}(2 - d) - 1 - z & r - 1 -z
    \end{pmatrix}.
\end{equation}
\end{document}
